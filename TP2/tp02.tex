\documentclass{sig-alternate}

\begin{document}

\title{Seguimiento de un blanco m\'{o}vil}

\numberofauthors{2}

\author{
    \alignauthor
    Villa Fern\'{a}ndez, Santiago\\
    \email{svillafe@alu.itba.edu.ar} \\
    \ \\
    Gomez Vidal, Maximiliano\\
    \email{dgomezvi@alu.itba.edu.ar} \\
    \alignauthor
    Sessa, Carlos\\
    \email{csessa@alu.itba.edu.ar} \\
    \ \\
    Abramowicz, Pablo\\
    \email{pabramow@alu.itba.edu.ar} \\
}

\maketitle

\begin{abstract}
En este art\'{i}culo se modela un sistema de seguimiento de blancos a lazo 
cerrado. Se analiza como influyen los distintos tipos de controladores: 
proporcional, integral y derivativo.
\end{abstract}

\keywords{Seguimiento de blancos, sistema de control, controlador proporcional,
controlador derivativo, controlador integral}

\section{Introducci\'{o}n}\label{introduccion}
Un sistema de seguimiento de blancos consiste en un radar y una antena.
Se desea que la antena apunte al blanco, por lo que se debe ajustar su 
posici\'{o}n  seg\'{u}n corresponda. Para esto se incluye un controlador que 
compara el \'{a}ngulo de la antena con el \'{a}ngulo en donde se encuentra el 
objetivo y proporciona el torque correspondiente para minimizar la discrepancia 
entre ambos.
%TODO: decir que hay en cada seccion

\section{Modelo}\label{modelo}
%TODO: explicar modelo

\section{Simulaciones}\label{simulaciones}
%TODO: explicar simulaciones

\subsection{Controlador proporcional}\label{proporcional}

\subsection{Controlador integral}\label{integral}

\subsection{Controlador derivativo}\label{derivativo}

\section{Comparaci\'{o}n de resultados}\label{resultados}
%TODO: comparar resultados, quizas vuela despues...

\section{Conclusiones}\label{conclusiones}
%TODO: conclusiones buena onda

\end{document}