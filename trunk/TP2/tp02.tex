\documentclass{sig-alternate}

\begin{document}

\title{Seguimiento de un blanco m\'{o}vil}

\numberofauthors{2}

\author{
    \alignauthor
    Villa Fern\'{a}ndez, Santiago\\
    \email{svillafe@alu.itba.edu.ar} \\
    \ \\
    Gomez Vidal, Maximiliano\\
    \email{dgomezvi@alu.itba.edu.ar} \\
    \alignauthor
    Sessa, Carlos\\
    \email{csessa@alu.itba.edu.ar} \\
    \ \\
    Abramowicz, Pablo\\
    \email{pabramow@alu.itba.edu.ar} \\
}

\maketitle

\begin{abstract}
En este art\'{i}culo se modela un sistema de seguimiento de blancos a lazo 
cerrado. Se analiza como influyen los distintos tipos de controladores: 
proporcional, integral y derivativo.
\end{abstract}

\keywords{Seguimiento de blancos, sistema de control, controlador proporcional,
controlador derivativo, controlador integral}

\section{Introducci\'{o}n}\label{introduccion}
Un sistema de seguimiento de blancos consiste en un radar y una antena.
Se desea que la antena apunte al blanco, por lo que se debe ajustar su 
posici\'{o}n  seg\'{u}n corresponda. Para esto se incluye un controlador que 
compara el \'{a}ngulo de la antena con el \'{a}ngulo en donde se encuentra el 
objetivo y proporciona el torque correspondiente para minimizar la discrepancia 
entre ambos.

En la secci\'{o}n \ref{modelo} se describe el modelo del sistema. En la
secci\'{o}n \ref{simulaciones} se realizan las simulaciones correspondientes
a los tres tipos de controladores utilizados: proporcional, integral y 
derivativo. Por \'{u}ltimo, se exponen las conclusiones en la secci\'{o}n
\ref{conclusiones}.

\section{Modelo}\label{modelo}
La din\'{a}mica de la antena se modela seg\'{u}n la ecuaci\'{o}n diferencial:
\begin{equation}
\label{dinamica_antena}
I \ddot\theta = - b \dot\theta + u(t)
\end{equation}
donde $\theta$ es el \'{a}ngulo que corresponde a la direcci\'{o}n en la que
apunta el radar, $I$ es el momento de inercia de la antena y $b$ es una 
constante positiva que vincula la fuerza viscosa que act\'{u}a sobre la antena.
El torque que producen los motores sobre la antena est\'{a} representado por 
la funci\'{o}n $u$.

Se desea que el sistema sea a lazo cerrado. Para esto se mide el \'{a}ngulo
$\theta$ de la antena y se lo compara con el \'{a}ngulo $\theta_{R}$ que marca
la ubicaci\'{o}n real del objetivo. La diferencia entre ambos constituye una
se\~{n}al de error $e(t) = \theta_{R}(t) - \theta(t)$ que se ingresa nuevamente 
al controlador.

\section{Simulaciones}\label{simulaciones}
%TODO: explicar simulaciones

\subsection{Controlador proporcional}\label{proporcional}

\subsection{Controlador integral}\label{integral}

\subsection{Controlador derivativo}\label{derivativo}

\section{Conclusiones}\label{conclusiones}
%TODO: conclusiones buena onda

\end{document}