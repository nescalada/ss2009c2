\documentclass{sig-alternate}

\begin{document}

\title{Simulando una cola simple}

\numberofauthors{2}

\author{
    \alignauthor
    Sessa, Carlos\\
    \email{csessa@alu.itba.edu.ar} \\
    \ \\
    Abramowicz, Pablo\\
    \email{pabramow@alu.itba.edu.ar} \\
    \ \\
    \alignauthor
    Gomez Vidal, Maximiliano\\
    \email{dgomezvi@alu.itba.edu.ar} \\
    \ \\
    Villa Fern\'{a}ndez, Santiago\\
    \email{svillafe@alu.itba.edu.ar}
}

\maketitle

\begin{abstract}
En este art\'{i}culo de simula un sistema de cola simple y se estiman
par\'{a}metros tales como la longitud media de la cola y el tiempo de
atenci\'{o}n. Se analiza tambi\'{e}n el costo operativo asociado.
\end{abstract}

\keywords{Modelo de cola simple, FIFO, tiempo de espera, sistema de atenci\'{o}n
al cliente, servidor simple}

\section{Introducci\'{o}n}\label{introduccion}

Las colas se utilizan en sistemas inform\'{a}ticos, transportes y operaciones 
de investigaci\'{o}n, entre otros. Los objetos, personas o eventos son tomados 
como datos que se almacenan para su posterior procesamiento.

La cola simple es el sistema m\'{a}s sencillo de analizar mediante una
simulaci\'{o}n por eventos discretos.

El sistema de cola simple se encuentra caracterizado principalmente por el 
tiempo necesario para atender a un cliente y la tasa de llegada de los mismos.

En la secci\'{o}n \ref{modelo} se describe el modelo de cola simple y los
par\'{a}metros involucrados. En la secci\'{o}n \ref{simulacion} se realizan
las simulaciones, se estudia el comportamiento del sistema y se analiza
el costo operativo resultante. Finalmente, se exponen las conclusiones en 
la secci\'{o}n \ref{conclusiones}.

%TODO modelo de cola simple
\section{Modelo de cola simple}\label{modelo}

%TODO simulaciones
\section{Simulaci\'{o}n}\label{simulacion}

%TODO conclusiones
\section{Conclusiones}\label{conclusiones}

\end{document}
