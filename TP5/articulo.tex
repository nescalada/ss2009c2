%\documentclass[a4paper,10pt]{amsart}
\documentclass[a4paper,10pt]{article}
\usepackage[latin1]{inputenc}
\usepackage[spanish]{babel}
\usepackage[dvips]{epsfig}
\usepackage{graphics}
\usepackage{verbatim}
\usepackage{moreverb}
\title{Trabajo Practico Final de Simulaci\'on de Sistemas(72.25)}

%\author[Abramowicz P., Gomez Vidal M., Sessa C., Villa Fernandez S.]{
\author{
Abramowicz Pablo\\
Gomez Vidal Maximiliano\\
Sessa Carlos\\
Villa Fernandez Santiago\\
}

\begin{document}
\maketitle

\section*{\underline{Item a}}
\section*{\underline{Item b}}
\section*{\underline{Item c}}
%Indicar los tipos de eventos y el espacio de eventos
Como se observo las colas involucradas son $S = \{R, E1, E3, OFT, PSF, E2, C\}$.\\
Todas las colas del sistema se modelan con capacidad infinita y un comportamiento \textbf{FIFO} (\textit{first in- first out}), salvo la cola $E3$ 
donde se considera que los clientes que todav\'ia no alcanzan a llenar el formulari\'o, son desplazados un lugar hacia atras en dicha cola.\\
Los eventos para cada cola del sistema son la llegada y la salida de un cliente.
Utilizando subindices para nombrar los eventos de cada cola, el espacio de eventos ($E$) del sistema resulta:\\
$$E = \{Ra , Rd , E1a , E1d , E3a , E3d , OFTa , OFTd , PSFa , PSFd , E2a , E2d , Ca , Cd \}$$\\
Donde el subindice \textit{a} indica la llegada de un cliente a la cola y el subindice \textit{d} indica la partida de un cliente de la cola.

\section*{\underline{Item d}}
\section*{\underline{Item e}}
\section*{\underline{Item f}}
\section*{\underline{Item g}}



\end{document}
