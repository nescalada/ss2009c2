\documentclass{sig-alternate}

\begin{document}

%TODO: cambiar este título horrible
\title{L'Ecuyer y velocidad WARP}

\numberofauthors{2}

\author{
    \alignauthor
    Gomez Vidal, Maximiliano\\
    \email{dgomezvi@alu.itba.edu.ar} \\
    \ \\
    Abramowicz, Pablo\\
    \email{pabramow@alu.itba.edu.ar} \\
    \ \\
    \alignauthor
    Sessa, Carlos\\
    \email{csessa@alu.itba.edu.ar} \\
    \ \\
    Villa Fern\'{a}ndez, Santiago\\
    \email{svillafe@alu.itba.edu.ar}
}

\maketitle

\begin{abstract}
En este art\'{i}culo se analiza el generador de n\'{u}meros pseudo-aleatorios
sugerido por L'Ecuyer y se lo pone a prueba utilizando el test de 
Kolmogorov-Smirnov y el test $\chi^{2}$. Se realiza tambi\'{e}n una simulaci\'{o}n
del tiempo de vuelo de una nave espacial mediante el m\'{e}todo de Montecarlo.
\end{abstract}

\keywords{L'Ecuyer, generador lineal congruencial, simulaci\'{o}n de Montecarlo,
 test $\chi^{2}$, test Kolmogorov-Smirnov, propulsi\'{o}n WARP}

%TODO
\section{Introducci\'{o}n}\label{introduccion}

%TODO
\section{Generador de L'Ecuyer}\label{generador}

%TODO
\subsection{Descripci\'{o}n}\label{descripciongenerador}

%TODO
\subsection{Test $\chi^{2}$}\label{testchicuadrado}

%TODO
\subsection{Test Kolmogorov-Smirnov}\label{testks}

%TODO
\subsection{Densidad triangular}\label{triangular}

%TODO
\section{Sistema de propulsi\'{o}n WARP}\label{simulacionpropulsor}

%TODO
\section{Conclusiones}\label{conclusiones}

\end{document}
