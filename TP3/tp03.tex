\documentclass{sig-alternate}

\begin{document}

%TODO: cambiar este título horrible
\title{L'Ecuyer y velocidad WARP}

\numberofauthors{2}

\author{
    \alignauthor
    Gomez Vidal, Maximiliano\\
    \email{dgomezvi@alu.itba.edu.ar} \\
    \ \\
    Abramowicz, Pablo\\
    \email{pabramow@alu.itba.edu.ar} \\
    \ \\
    \alignauthor
    Sessa, Carlos\\
    \email{csessa@alu.itba.edu.ar} \\
    \ \\
    Villa Fern\'{a}ndez, Santiago\\
    \email{svillafe@alu.itba.edu.ar}
}

\maketitle

\begin{abstract}
En este art\'{i}culo se analiza el generador de n\'{u}meros pseudo-aleatorios
sugerido por L'Ecuyer y se lo pone a prueba utilizando el test de 
Kolmogorov-Smirnov y el test $\chi^{2}$. Se realiza tambi\'{e}n una simulaci\'{o}n
del tiempo de vuelo de una nave espacial mediante el m\'{e}todo de Montecarlo.
\end{abstract}

\keywords{L'Ecuyer, generador lineal congruencial, simulaci\'{o}n de Montecarlo,
 test $\chi^{2}$, test Kolmogorov-Smirnov, propulsi\'{o}n WARP}

\section{Introducci\'{o}n}\label{introduccion}

El concepto de aleatoriedad est\'{a} presente en diversos campos de la ciencia,
tales como la criptograf\'{i}a y la estad\'{i}stica.

En determinadas ocasiones es deseable generar secuencias de n\'{u}meros
aleatorios. Resulta l\'{o}gico pensar en generar dichas secuencias utilizando
una computadora. Sin embargo, los n\'{u}meros al azar surgen \'{u}nicamente
de procesos naturales que no pueden ser recreados en un dispositivo determinista
como lo es una computadora. La soluci\'{o}n a este problema consiste en utilizar
algoritmos que permitan generar secuencias suficientemente largas de n\'{u}meros
que presenten una distribuci\'{o}n estad\'{i}stica regular. Es decir, que no debe
ser posible distinguir mediante pruebas estad\'{i}sticas que la secuencia no
ha sido construida realmente al azar.

Una de los generadores m\'{a}s conocidos de n\'{u}meros pseudo-aleatorios es
el generador lineal congruencial (LCG). 

En la secci\'{o}n \ref{generador} se 
analiza el geneador de L'Ecuyer, que utiliza dos LCG para generar las secuencias.
En la secci\'{o}n \ref{pruebagenerador} se somete el generador de L'Ecuyer
a tests estad\'{i}sticos para analizar la distribuci\'{o}n de los n\'{u}meros
generados. En la secci\'{o}n \ref{triangular} se obtiene una distribuci\'{o}n
triangular a partir de la salida del generador. En la secci\'{o}n
\ref{simulacionpropulsor} se realiza una simulaci\'{o}n del tiempo de vuelo
de la nave USS Enterprise utilizando el m\'{e}todo de Montecarlo. Finalmente,
se exponen las conclusiones en la secci\'{o}n \ref{conclusiones}.

%TODO
\section{Generador de L'Ecuyer}\label{generador}

El generador sugerido por L'Ecuyer combina dos LCG de acuerdo al siguiente 
algoritmo:

\textit{PASO 1}
Seleccionar una semilla $X_{1,0}$ en el rango $[1, 2147483562]$ para el LCG$1$
y $X_{2,0}$ en el rango $[1, 2147483398]$ para el LCG$2$.

\textit{PASO 2}
Evaluar cada generador individual

\begin{equation}
\label{LCG1}
X_{1,n+1} = 40014 \ X_{1,n} \ mod \ 2147483563
\end{equation}

\begin{equation}
\label{LCG2}
X_{2,n+1} = 40692 \ X_{2,n} \ mod \ 2147483399
\end{equation}

\textit{PASO 3}
Computar

\begin{equation}
\label{xn}
X_{n+1} = (X_{1,n+1} - X_{2,n+1}) \ mod \ 2147483562
\end{equation}

\textit{PASO 4}
Computar

\begin{equation}
\label{un}
U_{n+1} = 
    \begin{cases}
    \frac{X_{n+1}}{2147483563}, X_{n+1} > 0\\
     \ \\
    \frac{2147483562}{2147483563}, X_{n+1} = 0\\
    \end{cases}
\end{equation}

\textit{PASO 5}
Hacer $n = n + 1$ e ir al \textit{PASO 2}.

\section{Probando el generador de L'Ecuyer}\label{pruebagenerador}

\subsection{Test $\chi^{2}$}\label{testchicuadrado}

%TODO
\subsection{Test Kolmogorov-Smirnov}\label{testks}

%TODO
\section{Densidad triangular}\label{triangular}

%TODO
\section{Sistema de propulsi\'{o}n WARP}\label{simulacionpropulsor}

%TODO
\section{Conclusiones}\label{conclusiones}

\end{document}
