\documentclass{sig-alternate}

\begin{document}

\title{Simulando una cola simple}

\numberofauthors{2}

\author{
    \alignauthor
    Sessa, Carlos\\
    \email{csessa@alu.itba.edu.ar} \\
    \ \\
    Abramowicz, Pablo\\
    \email{pabramow@alu.itba.edu.ar} \\
    \ \\
    \alignauthor
    Gomez Vidal, Maximiliano\\
    \email{dgomezvi@alu.itba.edu.ar} \\
    \ \\
    Villa Fern\'{a}ndez, Santiago\\
    \email{svillafe@alu.itba.edu.ar}
}

\maketitle

\begin{abstract}
En este art\'{i}culo de simula un sistema de cola simple y se estiman
par\'{a}metros tales como la longitud media de la cola y el tiempo de
atenci\'{o}n. Se analiza tambi\'{e}n el costo operativo asociado.
\end{abstract}

\keywords{Modelo de cola simple, FIFO, tiempo de espera, sistema de atenci\'{o}n
al cliente, servidor simple}

\section{Introducci\'{o}n}\label{introduccion}

Las colas se utilizan en sistemas inform\'{a}ticos, transportes y operaciones 
de investigaci\'{o}n, entre otros. Los objetos, personas o eventos son tomados 
como datos que se almacenan para su posterior procesamiento.


La cola simple, denominada $M/M/1/\infty/FIFO$ o simplemente $M/M/1$, es el 
sistema m\'{a}s sencillo de analizar mediante una simulaci\'{o}n por eventos 
discretos.


El sistema de cola simple se encuentra caracterizado principalmente por el 
tiempo necesario para atender a un cliente y la tasa de llegada de los mismos.


En la secci\'{o}n \ref{modelo} se describe el modelo de cola simple y los
par\'{a}metros involucrados. En la secci\'{o}n \ref{simulacion} se realizan
las simulaciones, se estudia el comportamiento del sistema y se analiza
el costo operativo resultante. Finalmente, se exponen las conclusiones en 
la secci\'{o}n \ref{conclusiones}.

\section{Modelo de cola simple}\label{modelo}

Este modelo asume que los clientes llegan al sistema mediante un proceso de
Poisson con una tasa media de \\ $\lambda[clientes/hora]$. Tambi\'{e}n se asume
que el servidor atiende a cada uno de los clientes con un tiempo de servicio
distribuido en forma exponencial con media $1/\mu$.


Si cuando llega un nuevo cliente el servidor se encuentra libre, entonces el 
mismo es atendido en forma inmediata. Caso contrario, el cliente debe ingresar
en la cola y esperar su turno.


Resulta de inter\'{e}s poder establecer las probabilidades en estado estacionario,
cuando el sistema se encuentra en equilibrio:

\begin{equation}
\label{probabilidad_largo_plazo}
p_{n} = \lim_{t \rightarrow \infty} P \{ N_{t} = n \}
\end{equation}

donde $N_{t}$ es la cantidad de clientes en el sistema en el instante $t$.
Generalizando para el estado $n$ se obtiene una relaci\'{o}n de recurrencia
cuya soluci\'{o}n es:

\begin{equation}
\label{prob_estado_estacionario}
p_{n} = \left( \frac{\lambda}{\mu} \right)^{n} p_{0}
\end{equation}

con condiciones iniciales

\begin{equation}
\label{prob_estado_estacionario_ci}
p_{1} = \frac{\lambda}{\mu} p_{0}
\end{equation}


El factor $\rho = \lambda / \mu$ se denomina intensidad de tr\'{a}fico del
sistema. De esta forma la ecuaci\'{o}n \eqref{prob_estado_estacionario} puede
reescribirse como:

\begin{equation}
\label{prob_estado_estacionario_con_rho}
p_{n} = (1 - \rho) \rho^{n}
\end{equation}


De esta ecuaci\'{o}n se desprende que la intensidad de tr\'{a}fico debe
cumplir $\rho < 1$ para que el sistema sea estable. El caso $\rho > 1$
implica que el servidor tarda m\'{a}s tiempo en atender a un cliente que nuevos
clientes en arribar al sistema, generando as\'{i} una cola cada vez m\'{a}s 
larga.


A partir de las ecuaciones obtenidas es posible describir la longitud media
de la cola $L_{q}$ y el n\'{u}mero medio de clientes en el sistema $L$ en 
funci\'{o}n de $\rho$:

\begin{equation}
\label{longitud_media_en_la_cola}
L_{q} = \frac{\rho^{2}}{1-\rho}
\end{equation}

\begin{equation}
\label{numero_medio_clientes_en_el_sistema}
L = \frac{\rho}{1-\rho}
\end{equation}

%TODO citar a este autor
%D.C. Little
%A Proof for the Queuing Formula L = W , Operations Research, 9, 1961, pp.383­-387)

El modelo no estar\'{i}a completo sin el an\'{a}lisis del tiempo medio de 
espera, medida importante en lo que respecta a la performance del sistema
de colas. El tiempo medio de un individuo en el sistema $W$ y en cola $W_{q}$
resultan:

\begin{equation}
\label{tiempo_medio_en_el_sistema}
W = \frac{1}{\mu - \lambda}
\end{equation}

\begin{equation}
\label{tiempo_medio_en_la_cola}
W_{q} = \frac{\rho}{\mu - \lambda}
\end{equation}


%TODO simulaciones
\section{Simulaci\'{o}n}\label{simulacion}

%TODO conclusiones
\section{Conclusiones}\label{conclusiones}

En un sistema de cola simple la estabilidad depende exclusivamente de la
intensidad de tr\'{a}fico.

\end{document}
